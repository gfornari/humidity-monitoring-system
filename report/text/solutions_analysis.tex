\section{Solutions analysis}
\label{sec:solutions_analysis}

There are some solutions in the literature that try to solve the exposed problem and some of them are very similar to each other.
They usually exploit sensors able to perceive humidity or temperature and the data about these informations are sent to a receiver that is provided with an Internet connection (or another type of connection) in order to communicate with a central server.

For example in \cite{7969984} the system proposed is based on some little wireless sensors (even more in one single room) that communicate with receivers (usually one for every room) containing an Arduino-based controller board that allows them to connect to the Internet.
Because of the very low dimension of the sensors, they are provided with a very little battery and a consequence of this choice is that in order to grant a long operative life of the system, the wireless range of the antenna must be quite short and the data have to be transmitted with long intervals between every dispatch.
