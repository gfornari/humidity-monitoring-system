\section{Solutions analysis}
\label{sec:solutions_analysis}

There are some solutions in the literature that try to solve the exposed problem and some of them are very similar to each other.
They usually exploit sensors able to perceive humidity or temperature and the data about these informations are sent to a receiver that is provided with an Internet connection (or another type of connection) in order to communicate with a central server.

For example in \cite{7969984} the system proposed is based on small wireless sensors (many in a single room) that communicate with receivers (usually one for every room) containing an Arduino-based controller board that allows them to connect to the Internet.
Because of the small size of the sensors (diameter of about 35 mm and thickness of about 20 mm), they are provided with a very little battery and a consequence of this choice is that in order to grant a long operative life of the system, the wireless range of the antenna (between 10 m and 30 m, depending on the physical obstacles) must be quite short and the data have to be transmitted with long intervals between every dispatch.

Another interesting deployment is exposed in \cite{6348392}, in which the general structure is very similar to the one described previously, with the difference that a Wireless Sensor Network is exploited.
This means that the sensor nodes can communicate, as well as with a control unit (called gateway in the paper) connected to the Internet, to each other.
In order to do this, the sensors use the ZigBee protocol. However, this may not be the best choice if the desired solution should be cheap.

In both of the works, the data retrieved by the receivers are then sent to a server (or, more in general, a cloud architecture) and made available to some clients, browser-based and app-based (for mobile) in the former, only browser-based in the latter. 