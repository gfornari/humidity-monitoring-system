\section{Introduction}
\label{sec:introduction}

% Introduzione allo scenario spiegando la questione delle opere d'arte, della
% loro necessità di stare in ambienti controllati. Inoltre, descrivere le
% caratteristiche desiderabili del sistema e perché (prima il perché e poi il
% cosa), tipo:

% Gli ambienti che necessitano di un monitoraggio di umidità o di temperatura
% spesso sono edifici storici come chiese, ville o musei al cui interno
% risiedono delle opere d'arte che si vuole preservare. Può risultare
% problematico l'ultilizzo di cavi per la loro invasività sia dal punto di vista
% architetturare (come il passaggio di cavi all'interno di murature dove non è
% prevista l'installazione di un tale impianto), sia dal punto di vista
% artistico poiché può disturbare la fruzione del patrimonio artistico. Per
% questo motivo, il sistema deve utilizzare il più possibile connessioni
% wireless.

% Alcune caratteristiche desiderabili sono: sensori wireless, sensori con
% batteria a lunga durata (anni), facilità di installazione, accesso facile ai
% dati per monitoraggio, visualizzazione di eventi critici, economicità,
% possibilimente riuso di componenti già installate.

% Trarre ispirazione dal doc inviato a palazzi, sezione Introduzione e Analisi.
% In pratica, trasformare quanto fatto in caratteristiche desiderabili. Spunto
% può essere preso anche dai paper citati in questo documento.
