\section{Our solution}
\label{sec:our_solution}

% Introduzione alla nostra soluzione descrivendo il sistema ad alto livello e
% riprendendo quanto detto sulle caratteristiche desiderabili dell'introduzione
% (per quanto possibile qui). Nota: in tutto il doc dovremmo fare riferimento
% alle caratteristiche descritte in introduzione e dire come le abbiamo assolte.

% Anche per questa sezione, prendere spunto dal doc inviato a Palazzi.

% In questa introduzione alla nostra soluzione si arriverà a dire che si possno
% identificare 4 componenti principali del nostro sistema (praticamente quello
% che propone Giulio):
% - sensori wireless
% - gateway (controller unit)
% - database
% - interfaccia

% Inserire un diagramma architetturale di alto livello.

% Giusto per traccia (da NON seguire), la suddivisione delle componenti
% principali può anche essere su base funzionale, quindi:
% - rilevazione dei dati da parte dei sensori e loro trasmissione alle
%   centraline
% - salvataggio dei dati da parte della centralina e loro sincronizzazione con
%   un database remoto
% - dashboard e visualizzazione dei dati e recenti eventi critici

\subsection{Wireless sensors nodes}
% Prendo spunto da quanto proposto da Giulio.

% Parlare:
% - delle componenti dei sensori e delle caratteristiche (batteria, sensore
%   umidità, sensore temperatura, tipo di segnale)
% - del reverse engineering del protocollo (come ci siamo arrivati)
% - della specifica del protocollo di comunicazione (a cosa siamo arrivati)


\subsection{Gateway}
% Parlare:
% - delle componenti del gateway e delle caratteristiche (collegato alla
%   corrente ma possibilità di mettere una batteria, connessione internet,
%   antenne)
% - di come avviene la rilevazione di nuovi sensori
% - di come gestisce la ricezione del segnale (quindi rispetto al protocollo
%   descritto prima)
% - di come gestisce la riconnessione ad internet
% - di cosa fa quando ha la misurazione decodificata (dove salva i dati in
%   locale e sincronizzazione)


\subsection{Database}
% Parlare:
% - di quale database (quindi tipo di database e infrastruttura cloud)
% - della struttura dei dati (obiettivo della struttura: flessibilità ed
%   estensibilità)


\subsection{User interface}
% Parlare:
% - del fatto che (cit.) è Web e che si interfaccia direttamente al db
% - di cosa permette di visualizzare (dati recenti divisi per building ed eventi
%   critici divisi)
